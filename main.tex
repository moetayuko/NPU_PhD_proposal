% ! TEX program = xelatex
\documentclass{nwpuphdproposal}

\usepackage{mathtools}

\school{计算机学院}
\author{论外}
\studentnumber{1145141980}
\major{计算机科学与技术}
\advisor{李田所}
\date{1919年8月10日}

\addbibresource{ref.bib}

\begin{document}

\maketitle

\begin{nomargin}
	\zihao{5}%
	\SetTblrInner{vspan=even}%
	\begin{tblr}{X[c]|X[c]|X[c]|X[c]|X[c]}
		\textbf{论文题目} & \SetCell[c=4]{c} 论日本食用水水质解决方案 &&&                                                  \\\hline
		\textbf{开题次数} & \SetCell[c=4]{c} \Checkedbox 第一次\hspace*{3.5em}$\Box$ 第二次 &&&         \\\hline
		\SetCell[r=2]{c}{\setlength{\baselineskip}{0pt}\bfseries 论文类型                                               \\(请在有关项目\\下作$\surd$记号)} & 基础研究 & 应用研究 & 工程技术 & 跨学科研究 \\\hline
		              & $\surd$                                                       &  &  & \\
		\hline
	\end{tblr}
\end{nomargin}

\section{学位论文研究依据}

\secdesc*{学位论文的选题依据和研究意义,国内外研究现状和发展态势,主要参考文献,以及已有的工作积累和研究成果。}

\subsection{选题依据和研究意义}

\subsection{国内外研究现状}

\subsubsection{GNN}

参考文献直接cite即可\cite{DBLP:conf/nips/YouYL20},默认使用GB/T 7714-2015格式。

\subsection{已有工作积累和研究成果}

\subsubsection{已发表论文}

自己的工作填写在单独的bib文件里,不会出现在参考文献列表。可在bib里加粗自己名字和添加注解。
\begin{refsection}[mywork.bib]
	\nocite{*}
	\printbibliography[heading=none]
\end{refsection}

\printbibliography[title=主要参考文献]

\clearpage

\section{学位论文研究内容}
\secdesc*{学位论文的研究目标、研究内容及拟解决的关键性问题(可续页)}

\clearpage

\section{学位论文研究计划及预期目标}
\secdesc{拟采取的主要理论、研究方法、技术路线和实施方案(可续页)}

\subsubsection{基于xxx的xxx算法}

\hruleinbox
\secdesc{研究计划可行性,研究条件落实情况,可能存在的问题及解决办法(可续页)}
\setcounter{subsection}{2}

\clearpage

\secdesc{研究计划及预期成果}
\smallskip
\begin{nomargin}[h]
	\zihao{5}%
	\begin{tblr}{ c|c|X[c] }
		\hline
		\SetCell[r=5]{c}{\setlength{\baselineskip}{20pt}研\\究\\计\\划} & 起止年月 & 完成内容 \\\hline
		 &  &                       \\\hline
		 &  &                       \\\hline
		 &  &                       \\\hline
		 &  &                       \\\hline
	\end{tblr}
\end{nomargin}

\secdesc*{预期创新点及成果形式}
\vfill

\begin{nomargin}[h]
	\begin{tblr}{colspec={Q[c,wd=1.5cm]|X[c]}, rowspec={m{1cm}} }
		\hline
		\textbf{备注} & \\
	\end{tblr}
\end{nomargin}

% \clearpage

\end{document}

% vim: ts=2 sw=2
